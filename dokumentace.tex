%% LyX 2.2.3 created this file.  For more info, see http://www.lyx.org/.
%% Do not edit unless you really know what you are doing.
\documentclass[12pt,czech,pdftex,titlepage]{report}
\usepackage{lmodern}
\usepackage{lmodern}
\renewcommand{\familydefault}{\ttdefault}
\usepackage[T1]{fontenc}
\usepackage[utf8]{inputenc}
\usepackage[a4paper]{geometry}
\geometry{verbose,tmargin=2cm,lmargin=2cm,rmargin=3cm}
\usepackage{color}
\usepackage{array}
\usepackage{float}
\usepackage{graphicx}

\makeatletter

%%%%%%%%%%%%%%%%%%%%%%%%%%%%%% LyX specific LaTeX commands.
\newcommand{\noun}[1]{\textsc{#1}}
%% Because html converters don't know tabularnewline
\providecommand{\tabularnewline}{\\}

%%%%%%%%%%%%%%%%%%%%%%%%%%%%%% User specified LaTeX commands.


\usepackage[czech]{babel}
\usepackage{textcomp}
\usepackage{amsfonts}
\usepackage{titlesec}


\titleformat{\chapter}
  {\normalfont\LARGE\bfseries}{\thechapter}{1em}{}
\titlespacing*{\chapter}{0pt}{0ex plus 1ex minus .2ex}{2.0ex plus .2ex}

\makeatother

\usepackage{babel}
\usepackage{listings}
\renewcommand{\lstlistingname}{\inputencoding{latin2}V�pis}

\begin{document}
\begin{titlepage} \vspace*{2cm}
 {\centering\includegraphics{pripravena-dokumentace-v-latexu/logo}\par }
\centering \vspace*{2cm}
 {\Large Semestrální práce z KIV/PC\par } \vspace{1.5cm}
 {\Huge\bfseries Vyhledávání cest v grafu 

technikou DFS\par } \vspace{2cm}
\par {\Large Radek Juppa}\medskip{}

\par {\Large {\small{}Student: A16B0039K} }

\vfill{}
\par {\Large 20.\,1.\,2018} \end{titlepage}

\thispagestyle{empty} \clearpage{}

\setcounter{page}{1}

\begin{table}
\begin{tabular}{>{\centering}p{4cm}>{\centering}p{10cm}>{\centering}p{4cm}}
 & %
\noindent\begin{minipage}[t]{1\columnwidth}%

\paragraph*{Obsah}
\begin{enumerate}
\item Zadání 
\begin{enumerate}
\item Specifikace výstupu programu 
\item Řazení výstupu 
\end{enumerate}
\item Analýza úlohy 
\begin{enumerate}
\item Analýza grafu 
\item Metoda procházení grafu 
\item Algoritmus
\end{enumerate}
\item Popis implementace 
\item Uživatelská příručka 
\item Závěr
\end{enumerate}
%
\end{minipage} & \tabularnewline
\end{tabular}

\end{table}

\pagebreak{}

\chapter*{1. Zadání}

\noindent Naprogramujte v ANSI C přenositelnou konzolovou aplikaci,
která bude procházet graf technikou DFS (Depth-First Search). Vstupem
aplikace bude soubor s popisem grafu. Výstupem je pak odpovídající
výčet všech cest mezi požadovanými uzly grafu. 
\begin{verbatim*}
Program se bude spouštět příkazem dfs.exe <soubor-grafu> <id1> <id2> <maxD>
\end{verbatim*}
\begin{itemize}
\item Symbol <soubor-grafu> zastupuje parametr – název vstupního souboru
se strukturou grafu. 
\item Následují identifikátory (dále jen id) dvou uzlů v grafu <id1> a <id2>,
mezi kterými bude spouštěn proces hledání cest. 
\item <maxD>je parametr popisující maximální délku cest, které mají být
nalezeny. 
\item Vámi vyvinutý program tedy bude vykonávat následující činnosti. 
\begin{enumerate}
\item Při spuštění bez potřebných parametru vypíše nápovědu pro jeho správné
spuštění a ukončí se.
\item Při spuštění s parametry načte zadaný vstupní soubor do vhodné struktury
reprezentující graf a mezi zadanými uzly najde všechny cesty, jejichž
délka nepřekročí konstantu nastavenou posledním parametrem.
\end{enumerate}
\end{itemize}

\subsection*{1.a) Specifikace výstupu programu}

Program bude na standardní výstup vypisovat jednotlivé cesty. Vždy
na jeden řádek právě jednu cestu. Cesty budou popsány posloupností
id jednotlivých uzlů oddělených pomlčkou (tj. znakem minus). Následovat
bude středník a popisky jednotlivých hran v grafu oddělené čárkou
(viz Příloha 1). Za posledním středníkem bude uvedena hodnota sekundární
metriky – relevance cesty, podle které budou cesty seřazeny (viz Řazení
výstupu).

\noindent \vspace{0.5cm}
Například tedy pro hledání cest mezi uzly A a B: 

\textit{\small{}A-B;h1;m1}{\small \par}

\textit{\small{}A-F-B;h2,h3;m2,3}{\small \par}

\textit{\small{}A-W-B;h4,h5;m4,5}{\small \par}

\textit{\small{}A-F-O-B;h2,h6,h7;m2,6,7}\medskip{}

\noindent Kde A,B,F,W,F,... jsou popisky uzlů grafu, hn jsou ohodnocení
hran a mc1,c2...cn je hodnota metrika relevance nalezené cesty.\pagebreak{}

\subsection*{1.b) Řazení výstupu}

Cesty budou na standardním výstupu seřazeny primárně podle jejich
délky. Pokud bude nalezeno více cest stejné délky, budou seřazeny
podle jejich relevance následujícím způsobem. Popisky v grafu nesou
informaci o kalendářním datu ve formátu YYYY-MM-DD. Každá cesta bude
tedy ohodnocena celým číslem, které bude odpovídat rozdílu v počtu
dní mezi nejstarším a nejnovějším datem, následně budou podle tohoto
čísla cesty se shodnou délkou seřazeny vzestupně.

\pagebreak{}

\chapter*{2. Analýza úlohy}

\section*{2.a) Analýza grafu}

Nejprve se pojďme podívat, co to je za graf. Na první pohled je vidět,
že je to velký graf s několika tisíci uzlů. Vizualizace celého grafu
by byla dost obtížná. Proto si vyberu jen nějaký podgraf. Použil jsem
k tomu \emph{Python} a knihovnu \emph{networkx}. Vybral jsem si prvních
16 různých vrcholů a vzal jsem v úvahu jen 8 sousedů každého vrcholu. Vizualizaci
tohoto podgrafu jsem provedl pomocí knihovny \emph{graphviz}.

\begin{figure}[H]
\includegraphics[scale=0.4]{fig0}\caption{podgraf}

\end{figure}

\noindent Na obrázku vidíme, že se jedná o orientovaný graf s kružnicemi. Z
těchto informací usuzuji, že obecný počet cest mezi 2 vrchly grafu,
vzhledem ke kružnicím, bude nekonečně mnoho. Proto dává smysl omezení
maximální délky cesty, které je uvedeno v zadání. 

\pagebreak{}

\section*{2.b) Metoda procházení grafu}

Pro prohledávání grafu existují 2 základní metody \emph{Breadth First
Search} - \emph{BFS} a \emph{Deep First Search} - \emph{DFS}. My máme
použít techniku DFS, což je prohledávání do hloubky. Konkrétně to
znamená, že při prohledávání postupujeme po listech do hloubky. Tedy
na obecném grafu, který je zobrazen na obrázku 2 to bude sekvence
vrcholů {[}1-2-5-7-8-3..{]} Kdežto BFS prochází nejdříve nejbližší
uzly. BFS sekvence by vypadala následovně: {[}1-2-3-4-6-5-7…{]}

\begin{figure}[H]

\includegraphics[scale=0.8]{fig01}\caption{DFS}

\end{figure}

\medskip{}
\pagebreak{}

\section*{2.c) Algoritmus}

Vraťme se teď opět k našemu grafu. Vezmu si ještě menší podgraf (Vyberu
podgraf s vrcholy: 1, 2, 4, 5, 6, 7, 8) a zkusím najít všechny cesty
mezi vrcholy 1 a 5 s maximální délkou cesty 2. viz obrázek 3

\begin{figure}[H]

\includegraphics[scale=0.15]{graphviz2}\caption{všechny cesty z 1 do 5 (maxD=2)}

\end{figure}

Zde je tabulka všech cest o maximální délce 2 z vrcholu 1:

\begin{table}[H]

\begin{tabular}{|c|c|c|c|}
\hline 
1-2-4 NE  & 1-3-5 \textbf{\textcolor{green}{\noun{ANO}}}  & 1-6-2 NE  & 1-7-5 \textbf{\textcolor{green}{ANO}} \tabularnewline
\hline 
1-2-5 \textbf{\textcolor{green}{ANO}}  & 1-4-2 NE  & 1-6-4 NE  & 1-7-8 NE \tabularnewline
\hline 
1-2-6 NE  & 1-4-6 NE  & 1-7-2 NE  & \tabularnewline
\hline 
1-2-8 NE  & 1-4-7 NE  & 1-7-3 NE  & \tabularnewline
\hline 
1-3-4 NE  & 1-4-8 NE  & 1-7-4 NE  & \tabularnewline
\hline 
\end{tabular}\caption{cesty z vrcholu 1 do vrcholu 5 (maxD = 2)}

\end{table}

Je vidět, že jen 3 cesty vedou do vrcholu 5. Budu tedy používat techniku
\emph{DFS}. Vrcholy, jimiž jsem prošel, si budu ukládat do zásobníku. Tak
budu postupovat dokud budu mít kam vstupovat nebo dokud nevstoupím
do cílového vrcholu nebo nedosáhnu maximální délky cesty. V opačném
případě se vrátím do vrcholu ze kteréhé jsem přišel. Pokud vstoupím
do cílového vrcholu cestu si uložím jako jedno z řešení. Na konci
pak všechny nalezené cesty seřadím a vytisknu.

\chapter*{3. Popis implementace }

Aby bylo možno graf procházet, je třeba ho nejprve uložit do nějaké
vhodné struktury. Samotný graf tvoří množina vrcholů a hran. Vrchol
je tvořen jednou integer hodnotou a hrana spojuje 2 vrcholy a je ohodnocena
řetězcem znaků. Vrcholy i hrany uložím do spojového seznamu.

\medskip{}

\noindent Struktura pro vrchol \emph{VERTEX} bude vypadat následovně: 

\inputencoding{latin2}\begin{lstlisting}[basicstyle={\scriptsize\ttfamily},tabsize=4]
typedef struct vertex{       // VERTEX (as linked list)     
    int id;                  // id of the vertex     
    EDGE *first_edge;        // reference to the first edge (as linked list)     
    struct vertex *next;     // next item 
} VERTEX;
\end{lstlisting}
\inputencoding{utf8}
\medskip{}

\noindent Hrany uložím do struktury \emph{EDGE}: 

\inputencoding{latin2}\begin{lstlisting}[basicstyle={\scriptsize}]
typedef struct edge{        // EDGE (as linked list)     
    int id;                 // id of vertex creating an edge with its parent     
    char *data;             // data(label) assign to the edge     
    struct edge *next;      // next item 
} EDGE;
\end{lstlisting}
\inputencoding{utf8}
\medskip{}
Tak, jak budu grafem procházet, budu si jednotlivé vrcholy ukládat
do zásovníku \emph{STACK}:

\inputencoding{latin2}\begin{lstlisting}[basicstyle={\scriptsize}]
typedef struct stack_item{     
    int id;                  // id of a vertex
    char *data;              // label of an edge
    struct stack_item *next; // reference to next item in linked list
} STACK_ITEM; 

typedef struct stack{     
    STACK_ITEM *top;         // the first vertex
    int count;               // count of items in the stack
} STACK;
\end{lstlisting}
\inputencoding{utf8}
\medskip{}
Zásobník je implementován ve svém vlastním modulu nazvaném \emph{stack.c.} Bude
ukládat aktuální cestu a bude k tomu používat dvě typické metody:

\inputencoding{latin2}\begin{lstlisting}[basicstyle={\scriptsize}]
void stack_push(STACK *stack, int id, char *data); 
int stack_pop(STACK *stack); 
\end{lstlisting}
\inputencoding{utf8}
\medskip{}
Pokud aktuální cesta dosáhne cílového vrcholu, bude uložena do seřazeného
spojového seznamu úspěšných cest typu \emph{RESULT}. Tuto strukturu
definuji následovně:

\inputencoding{latin2}\begin{lstlisting}[basicstyle={\scriptsize}]
typedef struct result{       // RESULT (as sorted linked list)
    int *vertexes;           // array of vertex ids
    struct tm *labels;       // array of labels (datetime struct)
    int length;              // length of path
    int score;               // number of days between the first and the last date
    struct result *next;     // next item
} RESULT;
\end{lstlisting}
\inputencoding{utf8}
\medskip{}
\pagebreak{}Pro uložení celkové struktury grafu včetně aktuální cesty
a výsledků bude sloužit struktura \emph{GRAPH}:

\inputencoding{latin2}\begin{lstlisting}[basicstyle={\scriptsize}]
typedef struct graph{     
    VERTEX *first_vertex;     
    STACK *current_path;     
    int count;     
    int start_vertex_id;     
    int target_vertex_id;     
    int limit;     
    RESULT *result; 
} GRAPH;
\end{lstlisting}
\inputencoding{utf8}
\medskip{}
Graf je implementován ve svém vlastním modulu nazvaném \emph{graph.c}
Nejdůležitejší metodou, která prochází graf a hledá cesty je metoda: 

\inputencoding{latin2}\begin{lstlisting}[basicstyle={\scriptsize}]
void graph_dfs(GRAPH *g)
\end{lstlisting}
\inputencoding{utf8}
\chapter*{4. Uživatelská příručka}

Program je dodán v podobě zdrojového kódu a proto je nutno jej nejprve
přeložit. Přeložení programu je shodné pro všechny hlavní platformy: Windows,
Linux, MacOS 

\medskip{}

\noindent Předpokladem úspěšného přeložení je přítomnost překladače
jazyka C. V kořenovém adresáři je soubor \emph{makefile}, který zajištuje
překlad. 

\noindent \medskip{}
Překlad se provede příkazem: \emph{make}

\inputencoding{latin2}\begin{lstlisting}[basicstyle={\scriptsize}]
$ make  
\end{lstlisting}
\inputencoding{utf8}
Příkaz make vypíše zprávu o spuštění překladače a vytvoří soubor \emph{dfs.exe}
V případě problému se nejprve přesvěčte, že compiler jazyka C funguje. Napište
příkaz pro vypsání verze překladače: \emph{gcc —version} Pokud nezobrazi
verzi, pak pravděpodobně není nainstalován nebo k němu chybí cesta
v proměnné \emph{\noun{PATH}}. Pokud se Vám program podařilo přeložit
a vytvořil se soubor \emph{dfs.exe}, pak jej můžete spustit následujícím
způsobem:

\inputencoding{latin2}\begin{lstlisting}[basicstyle={\scriptsize}]
$ ./dfs.exe <cesta> <id_start> <id_target> [maxDistance=5]
\end{lstlisting}
\inputencoding{utf8}
\medskip{}

\noindent Pokud nezadáte povinné vstupní parametry, program vypíše
nápovědu:

\inputencoding{latin2}\begin{lstlisting}[basicstyle={\scriptsize}]
$ ./dfs.exe  
***** Search for Path using DFS ****** 
* Seminar work of 'Programming in C' * 
*  Copyright (c) Radek Juppa, 2017   * 
**************************************  
Usage:  dfs.exe <filename> <id1> <id2> [maxD]  
Example:  
  dfs.exe graph.csv 1 2 3
\end{lstlisting}
\inputencoding{utf8}
\medskip{}

\noindent Příklad spuštění programu:

\inputencoding{latin2}\begin{lstlisting}[basicstyle={\scriptsize}]
$ ./dfs.exe sw2017-02-data.csv 1 29 3  

1-6-29;2007-06-14,2007-06-11;3 
1-25-29;2007-02-04,2007-02-01;3 
1-2-29;2007-02-16,2007-04-30;73 
1-3-29;2007-10-26,2008-01-18;84 
1-882-29;2007-11-02,2007-08-03;91 
1-4698-29;2007-05-29,2007-10-04;128 
1-21935-29;2007-01-31,2007-06-26;146  
1-780-29;2007-10-29,2007-05-30;152 
1-67-29;2007-08-07,2008-01-08;154 
1-7-29;2007-08-14,2007-02-23;172 
1-67-29;2007-05-08,2008-01-08;245 
1-25-29;2007-10-31,2007-02-01;272 
1-2-29;2008-02-15,2007-04-30;291 
1-3306-29;2008-01-17,2007-02-27;324 
1-777-29;2008-02-14,2007-03-13;338 
1-16-29;2008-03-05,2007-03-09;362 
1-2917-29;2008-09-09,2007-08-05;401 
1-1230-29;2007-01-15,2008-03-19;429
\end{lstlisting}
\inputencoding{utf8}
\medskip{}

\medskip{}
Program načetl soubor \emph{sw2017-02-data.csv} z aktuálního adresáře
a vyhledal všechny cesty mezi vrchly 1 a 29, které jsou dlouhé maximálně
3 hrany. Vidíme, že program našel 18 cest a vypsal je seřazené podle
jejich délky a hodnoty hran.

\chapter*{5. Závěr}

Tato úloha mi dala příležitost si vyzkoušet dynamické vytváření datovách
struktůr jako je spojový seznam, zásobník či graf pomocí jazyka C. Dále
jsem si vyzkoušel nástroj pro kontrolu uvolňování paměti programem
valgrind. A naučil jsem se napsat jednoduchý makefile pro různé platformy. Program
jsem testoval v prostředí Linuxu s překladačem gcc a v prostředí Windows
s překladačem MinGW. To považuji za solidní úvod do programování v
jazyce C. Dokumentaci jsem vytvořil v programu LYX.
\end{document}
